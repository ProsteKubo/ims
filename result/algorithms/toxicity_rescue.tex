% Pseudocode for toxicity detection and naloxone rescue
% Used in Chapter 3 and 5

\begin{algorithm}[H]
\caption{Toxicity Detection and Naloxone Rescue}
\begin{algorithmic}[1]
\Require Current state: $C(t)$, $C_e(t)$, petri\_state, naloxone\_enabled
\Ensure Patient status update
\State \Comment{Check toxicity threshold}
\If{$C(t) > C_{toxic}$} \Comment{e.g., 15 mg/L}
    \State \textbf{TOXICITY DETECTED}
    \If{naloxone\_enabled AND first\_overdose}
        \State \textbf{ADMINISTER\_NALOXONE:}
        \State \quad \Comment{Competitive antagonism - displace from receptors}
        \State \quad $C_e(t) \gets C_e(t) \cdot 0.1$ \Comment{90\% receptor displacement}
        \State \quad patient\_alive $\gets$ true \Comment{Rescue successful}
        \State \quad motivation $\gets 5.0$ \Comment{Severe withdrawal}
        \State \quad pain\_level $\gets 3$ \Comment{Maximum pain from withdrawal}
        \State \quad first\_overdose $\gets$ false
        \State \quad Log: "NALOXONE ADMINISTERED - Patient rescued"
    \Else
        \State \textbf{FATAL\_OVERDOSE:}
        \State \quad patient\_alive $\gets$ false \Comment{Deceased}
        \State \quad time\_of\_death $\gets t$
        \State \quad Stop() \Comment{Terminate simulation}
        \State \quad Log: "FATAL OVERDOSE - Simulation terminated"
    \EndIf
\EndIf
\State \Return patient\_status
\end{algorithmic}
\end{algorithm}

\vspace{1em}

\textbf{Naloxone Mechanism Notes:}
\begin{itemize}
    \item Naloxone is a competitive μ-opioid receptor antagonist with higher binding affinity than agonists
    \item The 90\% reduction in $C_e$ represents effective receptor displacement
    \item Physical $C(t)$ (plasma concentration) remains high - drug still in circulation
    \item Effect is rapid (2--5 minutes in reality, instantaneous in discrete event model)
    \item Creates severe withdrawal: precipitates pain, agitation, and strong motivation to redose
    \item Model limitation: Does not account for naloxone's own PK (short half-life ~30-90 min)
\end{itemize}
